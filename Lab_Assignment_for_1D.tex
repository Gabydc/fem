\documentclass[a4paper,10pt]{report}
\usepackage[utf8x]{inputenc}
\usepackage{amsmath}
\usepackage{amssymb}
\usepackage{amsfonts}
\usepackage{mathrsfs}
% \usepackage{natbib}
% \usepackage{graphicx} % figuras
% \usepackage[export]{adjustbox} % loads also graphicx
% \usepackage{float}
% \usepackage[font=footnotesize]{caption}
\usepackage{wrapfig}
% Title Page
\title{}
\author{}


\begin{document}

On the interval (0,1) we consider a steady-state convection-diffusion-reaction equation, with homogeneous Neumann boundary conditions:

\begin{align}\label{eq:1}
 -D\frac{d^2u}{dx^2}+\lambda u= f(x), \\
 -D\frac{du}{dx}(0)=0, \qquad -D\frac{du}{dx}(1)=0.
\end{align}
\begin{enumerate}
 \item{Assignment 1} Derive the $weak$ formulation. \\
 For this, we multiply Equation \eqref{eq:1} by the basis functions $\phi$ and integrate over the domain.\\
 \begin{equation}\label{eq:2}
 \int_{0}^{1} -D\phi\frac{d^2u}{dx^2}+\lambda \phi u dx= \int_{0}^{1}\phi f(x)dx 
\end{equation}
  \begin{align*}
 =\int_{0}^{1} -D\left[\frac{du}{dx}\left(\phi\frac{du}{dx}\right)-\frac{d\phi}{dx}\frac{du}{dx}+\lambda \phi u \right]dx=\\
 \int_{0}^{1} -D\mathbf{n}\cdot \phi\frac{du}{dx} ds+\int_{0}^{1}D\left[\frac{d\phi}{dx}\frac{du}{dx}+\lambda \phi u \right]dx
\end{align*}
But using Equation \eqref{eq:1} (bc) the first term above is zero, then we have:
 \begin{equation}\label{eq:3}
\int_{0}^{1}\left[D\frac{d\phi}{dx}\frac{du}{dx}+\lambda \phi u \right]dx= \int_{0}^{1}\phi f(x)dx
\end{equation}
\item{Assignment 2} Write the Galerkin formulation of the weak form as derived
in the previous assignment for a general number of elements given by n (hence
$x_n = 1$). Give the Galerkin equations, that is, the linear system in terms of
\begin{equation}\label{eq:4}
S\bar{u} = \bar{f},
 \end{equation}
all expressed in the basis-functions, $f(x)$, $\lambda$ and $D$.\\
For the Galerkin formulation we approximate ${u}$ with the basis functions $\phi_j$ as:
$$u(x)\sim \sum_{j=1}^na_j\phi_j(x).$$
Approximating $u$ and substituting $\phi=\phi_i$ in Equation \eqref{eq:3} we have:
 \begin{equation}\label{eq:5}
\sum_{j=1}^na_j\int_{0}^{1}\left[D\frac{d\phi_i}{dx}\frac{d\phi_j}{dx}+\lambda \phi_i \phi_j \right]dx= \int_{0}^{1}\phi_i f(x)dx
\end{equation}
Then 
\begin{align*}
S_{ij}^{e_k}&=a_j\int_{e_k}\left[D\frac{d\phi_i}{dx}\frac{d\phi_j}{dx}+\lambda \phi_i \phi_j \right]dx, \qquad &S_{ij}=\sum_{k=1}^{nel}S_{ij},\\
f_i^{e_k}&=\int_{e_k}\phi_i f(x)dx, \qquad &f_i=\sum_{k=1}^{nel}f_i^{e_k}.
\end{align*}
\item{Assignment 3} Write a matlab routine, called GenerateMesh.m that gener-
ates an equidistant distribution of meshpoints over the interval [0, 1], where
$x_1 = 0$ and $x_n = 1$ and $h =\frac{1}{n−1}$. You may use x = linspace(0,1,n).

\begin{table}[!h]\centering

\begin{tabular}{ |l| } 
\hline
\textbf{Algorithm 1}\\
\hline
\hspace{0.5cm}function [x]=GenerateMesh(n)\\
\hspace{1cm}x = linspace(0,1,n);\\
\hspace{0.5cm}end\\
\hline
\end{tabular}

\end{table}

\itemize{Assignment 4} Write a routine, called GenerateTopology.m, that generates
a two dimensional array, called elmat, which contains the indices of the ver-
tices of each element, that is
\begin{align*}
&elmat(i, 1) = i. \qquad for i ∈ {1, . . . , n − 1}.\\
&elmat(i, 2) = i + 1
\end{align*}
Next we compute the element matrix $S_{elem}$. In this case, the matrix is the
same for each element, that is, if we consider element $e_i$.

\begin{table}[!h]
\centering
\begin{tabular}{ |l| } 
\hline
\textbf{Algorithm 1}\\
\hline
\hspace{0.5cm}function [elmat] = GenerateTopology(n)\\
\hspace{1cm}elmat(i, 1) = i;\\
\hspace{1cm}elmat(i, 2) = i + 1;\\
\hspace{0.5cm}end\\
\hline
\end{tabular}
\end{table}
\item{Assignment 5} Compute the element matrix, Selem over a generic line ele-
ment $e_i$.
\end{enumerate}

\end{document}          
